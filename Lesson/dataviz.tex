\documentclass[10pt,aspectratio=169]{beamer}

\usetheme[titleformat=allcaps]{metropolis}
\usepackage{appendixnumberbeamer}

\usepackage{booktabs}
\usepackage[scale=2]{ccicons}

\usepackage{pgfplots}
\usepgfplotslibrary{dateplot}

\usepackage{xspace}
\newcommand{\themename}{\textbf{\textsc{metropolis}}\xspace}

\usepackage[rm,light]{roboto}
\usepackage[T1]{fontenc}
\usefonttheme{structurebold}
\setbeamerfont{title}{series=\bfseries\rmfamily,parent=structure}
\setbeamerfont{frametitle}{series=\bfseries\rmfamily,parent=structure}


\usepackage{xcolor}
\definecolor{material-indigo-900}{HTML}{1a237e}
\definecolor{material-bluegrey-900}{HTML}{263238}
\definecolor{material-bluegrey-600}{HTML}{546e7a}
\setbeamercolor{frametitle}{fg=white,bg=material-bluegrey-600}

\title{Scientific Visualization}
\date{Inria - University of Bordeaux}
\author{Nicolas P. Rougier -- Nicolas.Rougier@inria.fr}
\institute{\today}
% \titlegraphic{\hfill\includegraphics[height=1.5cm]{logo.pdf}}

\begin{document}

\maketitle


% -----------------------------------------------------------------------------
\begin{frame}{On the importance of vision}

  \begin{quote}
    ... about 50 percent of the cerebral cortex of primates is devoted
    exclusively to visual processing, and the estimated territory for humans is
    nearly comparable.
    \begin{flushright}
      \em \small The MIT Encyclopedia of the Cognitive Sciences
    \end{flushright}
  \end{quote}
\end{frame}


% -----------------------------------------------------------------------------
\begin{frame}{Anscombe's Quartet}
  \begin{columns}
    \begin{column}{.5\textwidth}    
      What is common to these data sets?\\
      ~\\
      \begin{tabular}{ll}
        Mean of x            & 9\\
        Sample variance of x & 11\\
        Mean of y            & 7.5\\
        Sample variance of y & 4.12\\
        Linear regression    & y=3.00+0.500*x\\
        R squared            & 0.666\\
        p value              & 0.0021\\
      \end{tabular}

    \end{column}
    \begin{column}{.5\textwidth}
      \includegraphics[width=.85\textwidth]{Anscombe-tables.png}
    \end{column}
  \end{columns}
%  \vfill
%  Data at \url{http://www.labri.fr/perso/nrougier/tmp/Anscombe.csv}\\

\end{frame}


% -----------------------------------------------------------------------------
\begin{frame}{Anscombe's Quartet}
  \begin{columns}
    \begin{column}{.5\textwidth}    
      What is common to these data sets?\\
      ~\\
      \begin{tabular}{ll}
        Mean of x            & 9\\
        Sample variance of x & 11\\
        Mean of y            & 7.5\\
        Sample variance of y & 4.12\\
        Linear regression    & y=3.00+0.500*x\\
        R squared            & 0.666\\
        p value              & 0.0021\\
      \end{tabular}

    \end{column}
    \begin{column}{.5\textwidth}
      \includegraphics[width=\textwidth]{Anscombe-plots.pdf}
    \end{column}
  \end{columns}

\vfill
\noindent A computer should make both calculations and graphs -- {\em Francis Anscombe (1918-2001)}
  
\end{frame}

%% % -----------------------------------------------------------------------------
%% \begin{frame}{Anscombe's Quartet}
%%   \begin{quote}
%%     {\Large A computer should make both calculations and graphs.}
%%     \begin{flushright}
%%       \em \small Francis Anscombe (1918-2001)
%%     \end{flushright}
%%   \end{quote}
%%   \vfill
%%   \begin{quote}
%%     {\Large The purpose of computing is insight, not numbers.}
%%     \begin{flushright}
%%       \em \small Richard Hamming, 1962
%%     \end{flushright}
%%   \end{quote}
%% \end{frame}



% -----------------------------------------------------------------------------
\begin{frame}{Cholera epidemic, London, 1854}
  \begin{columns}
    \begin{column}{.3\textwidth}
      John Snow (1813-1858) is considered one of the fathers of modern
      epidemiology, in part because of his work in tracing the source of a
      cholera outbreak in Soho, London, in 1854.
    \end{column}
    \begin{column}{.6\textwidth}
      \includegraphics[width=\textwidth]{cholera-old.png}
    \end{column}
  \end{columns}
\end{frame}

% -----------------------------------------------------------------------------
\begin{frame}{Cholera epidemic, London, 1854}
  \begin{columns}
    \begin{column}{.3\textwidth}
      John Snow (1813-1858) is considered one of the fathers of modern
      epidemiology, in part because of his work in tracing the source of a
      cholera outbreak in Soho, London, in 1854.
    \end{column}
    \begin{column}{.6\textwidth}
      \includegraphics[width=\textwidth]{cholera-modern.png}
    \end{column}
  \end{columns}
\end{frame}

% -----------------------------------------------------------------------------
\begin{frame}{What is data visualisation?}
  \begin{quote}
    Visualisation is a method of computing. It transforms the symbolic into the
    geometric, enabling researchers to observe their simulations and
    computations. Visualisation offers a method for seeing the unseen. It
    enriches the process of scientific discovery and fosters profound and
    unexpected insights.
    \begin{flushright}
      \em \small Visualisation in Scientific Computing, NSF report, 1987
    \end{flushright}
  \end{quote}
\end{frame}

% -----------------------------------------------------------------------------
\begin{frame}{The Visualization pipeline}
  \includegraphics[width=\textwidth]{pipeline.png}
  ~\\
  From Scalable Real-Time Visualization Using the Cloud,
  Holliman \& Watson, 2015.\\
\end{frame}


% -----------------------------------------------------------------------------
\begin{frame}{Quantitative vs Qualitative data}

  \textbf{Quantitative}
         (values or observations that can be measured)\\
  \begin{itemize}
  \item Continuous (e.g. temperature) 
  \item Discrete (e.g. number of inhabitants)
  \end{itemize}
  \vfill
  \textbf{Qualitative}
         (values or observations that can be sorted into groups or categories)\\
  \begin{itemize}
  \item Nominal (e.g. nationality)
  \item Ordinal (e.g. months)
  \item Interval (e.g. age groups)
  \end{itemize}
  
\end{frame}

% -----------------------------------------------------------------------------
\begin{frame}{Graphical elements}
  A scientific figure can be fully described by a set of graphic primitives
  with different attributes:
  \begin{itemize}
  \item Points, markers, lines, areas, ...
  \item Position, color, shape, size, orientation, curvature, ...
  \item Helpers, text, axis, ticks, ...
  \item Interaction, animation, ...
  \end{itemize}
  
  Questions is thus how to organize and link them to the underlying data.
\end{frame}

% -----------------------------------------------------------------------------
\begin{frame}{Principles of visual perception}
  \includegraphics[width=\textwidth]{gestalt.png}
\end{frame}

% -----------------------------------------------------------------------------
\begin{frame}{Visualization Analysis and Design (T. Munzner)}
  \begin{center}
    \includegraphics[width=.75\textwidth]{VAD.png}
  \end{center}
\end{frame}

% -----------------------------------------------------------------------------
\begin{frame}{Data Visualization catalogue (S. Recebba)}
  \begin{center}
    \href{https://datavizcatalogue.com/}{
    \includegraphics[width=\textwidth]{catalogue.png}}\\
  \end{center}
  {\tt datavizcatalogue.com}
\end{frame}


% -----------------------------------------------------------------------------
%% \begin{frame}{Chart Suggestions - A Thought-Starter (A. Abela)}
%%   \begin{center}
%%     \includegraphics[width=.75\textwidth]{chart.pdf}
%%   \end{center}
%% \end{frame}


  
% =============================================================================
\section{Ten simple rules for better figures}
% =============================================================================

% -----------------------------------------------------------------------------
\begin{frame}{Rule 1: Know your audience}
  \begin{center}
    \includegraphics[width=.75\textwidth]{rule-1.pdf}
  \end{center} 
\end{frame}

% -----------------------------------------------------------------------------
\begin{frame}{Rule 2: Identify your message}
  \begin{center}
    \includegraphics[width=.75\textwidth]{rule-2.pdf}
  \end{center} 
\end{frame}

% -----------------------------------------------------------------------------
\begin{frame}{Rule 3: Adapt the figure}
  \begin{center}
    \includegraphics[width=\textwidth]{rule-3.pdf}
  \end{center} 
\end{frame}

% -----------------------------------------------------------------------------
\begin{frame}{Rule 4: Captions are not optional}

  \begin{columns}
    \begin{column}{.45\textwidth}
        \begin{center}
          \includegraphics[width=.95\textwidth]{rule-4.png}
        \end{center}
    \end{column}
    \begin{column}{.45\textwidth}
        \begin{center}
          \includegraphics[width=.95\textwidth]{rule-4.png}
        \end{center}
    \end{column}
  \end{columns}
  \begin{columns}
    \begin{column}{.45\textwidth}
        \begin{center}
          Optical illusion
        \end{center}
    \end{column}
    \begin{column}{.45\textwidth}
        \begin{center}
          The A and B patches are actually the same color even
          though we perceived them at being different color.
        \end{center}
    \end{column}
  \end{columns}


\end{frame}

% -----------------------------------------------------------------------------
\begin{frame}{Rule 5: Do not trust the defaults}
  \includegraphics[width=\textwidth]{rule-5.pdf}
\end{frame}

% -----------------------------------------------------------------------------
\begin{frame}{Rule 6: Use color effectively}
  \includegraphics[width=\textwidth]{rule-6.pdf}
\end{frame}

% -----------------------------------------------------------------------------
\begin{frame}{Rule 6 bis: Above all, no jet. Ever.}
  \includegraphics[width=\textwidth]{rule-6bis.png}
\end{frame}

% -----------------------------------------------------------------------------
\begin{frame}{Rule 7: Do not mislead the reader}
  \includegraphics[width=\textwidth]{rule-7.pdf}
\end{frame}

\begin{frame}{Rule 7: Do not mislead the reader. Really.}
  \begin{center}
    \includegraphics[width=.75\textwidth]{rule-7bis.png}
  \end{center}
\end{frame}

% -----------------------------------------------------------------------------
\begin{frame}{Rule 8: Avoid chartjunk}
  \includegraphics[width=\textwidth]{rule-8.pdf}
\end{frame}

% -----------------------------------------------------------------------------
\begin{frame}{Rule 8 bis: Less is more}
  \begin{center}
    \includegraphics[width=.75\textwidth]{rule-8bis.png}
  \end{center}
\end{frame}

% -----------------------------------------------------------------------------
\begin{frame}{Rule 9: Message trumps beauty}
  \includegraphics[width=\textwidth]{rule-9.pdf}
\end{frame}

% -----------------------------------------------------------------------------
\begin{frame}{Rule 10: Get the right tool}
  \begin{itemize}
  \item PDFCrop (remove white borders)\\
    \url{http://pdfcrop.sourceforge.net}
  \item GraphViz (easy graph)\\
    \url{http://www.graphviz.org}
  \item ImageMagick (scripted image processing)\\
    \url{http://www.imagemagick.org/script/index.php}
  \item Gimp (bitmap image manipulation)\\
    \url{https://www.gimp.org}
  \item Inkscape (vector image manipulation)\\
    \url{https://www.inkscape.org}
  \item Tikz (scripted vector art)\\
    \url{http://www.texample.net/tikz/examples/all/}
  \item And many, many, many others …
  \end{itemize}
\end{frame}

% -----------------------------------------------------------------------------
\begin{frame}{Rule 10: Get the right tool}
    \includegraphics[width=\textwidth]{visualization-landscape.png}
\end{frame}


% -----------------------------------------------------------------------------
\begin{frame}{A note about formats}
  \begin{block}{Standard data formats}
    \vspace{0pt}
    \begin{itemize}
    \item[] \textbf{CSV} -- Comma-Separated Values
    \item[] \textbf{JSON} -- JavaScript Object Notation
    \end{itemize}
  \end{block}
  
  \begin{block}{Standard vector formats}
    \vspace{0pt}
    \begin{itemize}
    \item[] \textbf{PDF} -- Portable Document Format
    \item[] \textbf{SVG} -- Scalable Vector Graphics
    \end{itemize}
  \end{block}
  
  \begin{block}{Standard bitmap formats}
    \vspace{0pt}
    \begin{itemize}
    \item[] \textbf{PNG} -- Portable Network Graphics (lossless)
    \item[] \textbf{JPG} -- Joint Photographic Experts Group (lossy)
    \end{itemize}
  \end{block}

\end{frame}


% -----------------------------------------------------------------------------
\begin{frame}{Analyzing 1.1 Billion NYC Taxi and Uber Trips, with a Vengeance}

  \begin{columns}
    \begin{column}{.6\textwidth}
    The New York City Taxi \& Limousine Commission has released a staggeringly
    detailed historical dataset covering over 1.1 billion individual taxi trips
    in the city from January 2009 through June 2015. Taken as a whole, the
    detailed trip-level data is more than just a vast list of taxi pickup and
    drop off coordinates: it’s a story of New York.
    \begin{flushright}
      \small \textnormal Todd W. Schneider
                         (\href{https://toddwschneider.com/posts/analyzing-1-1-billion-nyc-taxi-and-uber-trips-with-a-vengeance/}{\tt toddwschneider.com})
    \end{flushright}
    \end{column}
    \begin{column}{.3\textwidth}
      \begin{center}
        \includegraphics[width=\textwidth]{taxi_pickups_map.png}
      \end{center}
    \end{column}
  \end{columns}
\end{frame}


% -----------------------------------------------------------------------------
\begin{frame}{The Simpsons by the Data}
  \begin{columns}
    \begin{column}{.5\textwidth}
      Analysis of 27 seasons of Simpsons data reveals the show’s most
      significant side characters, a pattern of patriarchy, declining TV
      ratings, and more
    \begin{flushright}
      \small \textnormal Todd W. Schneider
                         (\href{https://toddwschneider.com/posts/the-simpsons-by-the-data/}{\tt toddwschneider.com})
    \end{flushright}
    \end{column}
    \begin{column}{.4\textwidth}
      \begin{center}
        \includegraphics[width=\textwidth]{simpsons.png}
      \end{center}      
    \end{column}
  \end{columns}
\end{frame}


% -----------------------------------------------------------------------------
\begin{frame}{Conclusion}

  \begin{block}{Schedule}
    \begin{itemize}
    \item 15/11/2021 : Introduction + study
    \item 06/12/2021 : Dataviz catalogue + project
    \item 13/12/2021 : Layout and projection + project
    \item 10/01/2021 : Advanced concepts + project
    \item 24/01/2021 : Project
    \end{itemize}
  \end{block}

  \begin{block}{Project}
    \vspace{0pt} The goal of the project is to analyse wedding data
    (from INSEE) and to produce a one page PDF comparing data from
    2018 and 2019.  Your PDF must have at least 3 figures (with
    caption) and contains text to introduce your analysis.
  \end{block}
  
\end{frame}

  
\begin{frame}{References}
  \begin{block}{Books}
    \vspace{0pt}
    \begin{itemize}
    \item Scientific Visualization: Pyhpn + Matplotlib, N. Rougier, 2021
    \item Fundamentals of Data Visualization, C. Wilke, 2018
    \item Visualization Analysis and Design (\$), T. Munzner, 2014.
    \item Trees, maps, and theorems (\$), J.-L. Doumont, 2009.
    \item The Visual Display of Quantitative Information (\$), E.R. Tufte, 1983.
    \end{itemize}
  \end{block}

  \begin{block}{Other resources}
    \vspace{0pt}
    \begin{itemize}
    \item A Tour through the Visualization Zoo, J. Heer, M. Bostock, and V. Ogievetsky, 2010.
    \item The most misleading charts of 2015, fixed, K. Collins, 2015.
    \item Data is beautiful / reddit.
    \item \href{https://www.data-to-viz.com}{From data to viz} 
    \end{itemize}
  \end{block}
  
  %% \begin{block}{Bad examples (don't do that at home)}
  %%   \vspace{0pt}
  %%   \begin{itemize}
  %%   \item Junk charts, K. Fung, 2005-2016.
  %%   \item How to Display Data Badly, H. Wainer, 1984.
  %%   \end{itemize}
  %% \end{block}
  
\end{frame}





%% % =============================================================================
%% \section{Exercises}
%% % =============================================================================


%% % -----------------------------------------------------------------------------
%% \begin{frame}{Pixelating an image}

%%   How to obtain a pixelated picture from an original image?
  
%%   \begin{center}
%%     \includegraphics[width=.45\textwidth]{Hamilton.jpg}
%%     \includegraphics[width=.45\textwidth]{Hamilton-pixelated.jpg}
%%   \end{center}
%% \end{frame}


%% % -----------------------------------------------------------------------------
%% \begin{frame}{Misleading the reader}

%%   What's wrong with the graphic below? How would you fix it?\\
  
%%   \begin{center}
%%     \includegraphics[width=.75\textwidth]{obama.jpg}
%%   \end{center}

%% \end{frame}



%% % -----------------------------------------------------------------------------
%% \begin{frame}{Too much ink}
%%   \begin{columns}
%%     \begin{column}{.5\textwidth}
%%       Consider the figure on the right and try to remove as much ink as you can
%%       while keeping the most relevant information.\\
%%       ~\\
%%       Curve data at \url{http://www.labri.fr/perso/nrougier/tmp/curve.csv}\\
%%       ~\\
%%       Point data at \url{http://www.labri.fr/perso/nrougier/tmp/points.csv}\\
%%     \end{column}
%%     \begin{column}{.4\textwidth}
%%       \includegraphics[width=\textwidth]{too-much-ink.png}
%%     \end{column}
%%   \end{columns}
%% \end{frame}


%% % -----------------------------------------------------------------------------
%% \begin{frame}{Too much ink}
%%   \begin{columns}
%%     \begin{column}{.5\textwidth}
%%       Consider the figure on the right and try to remove as much ink as you can
%%       while keeping the most relevant information.\\
%%       ~\\
%%       Curve data at \url{http://www.labri.fr/perso/nrougier/tmp/curve.csv}\\
%%       ~\\
%%       Point data at \url{http://www.labri.fr/perso/nrougier/tmp/points.csv}\\
%%     \end{column}
%%     \begin{column}{.4\textwidth}
%%       \includegraphics[width=\textwidth]{too-much-ink-2.png}
%%     \end{column}
%%   \end{columns}
%% \end{frame}


\end{document}
